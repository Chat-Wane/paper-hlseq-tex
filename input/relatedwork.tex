
\section{Related Work}
\label{sec:relatedwork}
Current distributed collaborative editors use optimistic
replication~\cite{saito2002replication,saito2005optimistic} to provide
availability of data at a consistency cost. Optimistic replication designs
operations in two phases. First a replica prepares the result of an operation
and then broadcasts it to other collaborators. Second, remote replicas
integrate the newly received data.

Operational Transform approach (OT) and Conflict-free Replicated Data Type
approach (CRDT) belong to the optimistic class of replication. While CRDTs
share the computational cost of operations between the local and remote parts
of the replication protocol, the OT approaches prefer to offer free local
generation and pay an additional cost on the integration of remote operations.
Since the number of remote operations quadratically grows when the number of
collaborators increases, the CRDTs scale better than OT approaches.


We distinguish two classes of CRDTs for sequences. The tombstone based CRDT
approaches use death certificate on removed elements. Henceforth, these
elements are hidden to the users, but still remain in the underlying model and
undermine the performance. For instance, the number of tombstones in popular or
controversial pages in Wikipedia pages would be very high due to vandalism and
recoveries. The representatives of this class of CRDTs are
WOOT~\cite{oster2006data}, WOOTO~\cite{weiss2007wooki},
WOOTH~\cite{ahmed2011evaluating}, CT~\cite{grishchenko2010deep},
Treedoc~\cite{preguica2009commutative}, PPS~\cite{wu2010partial},
RGA~\cite{roh2011replicated} and \cite{Yu2012stringwise}.

To get rid of these marked elements, the tombstone based CRDTs require
additional protocols related to garbage collecting
mechanisms. In~\cite{letia2009crdts,roh2011replicated} they propose approaches
to safely purge the tombstones. However, it requires a global knowledge over
collaborators.  In~\cite{roh2011replicated}, they describe an approach that
obtains the global knowledge by maintaining a vector clock with one entry per
participant although it is too costly in large distributed network where
collaborators can enter and leave the system. In~\cite{letia2009crdts}, they
describe the \emph{core nebula} approach that permits to reach the consensus
over participants. However, this approach constrains the network topology to
make the consensus reachable and uses a costly catch up protocol.

The alternative to tombstones is the variable-size identifiers CRDT
approach. These CRDTs directly encode the total order of elements into their
unique identifiers. Thus, they do not require tombstones. However, the
identifiers can grow. In Logoot~\cite{weiss2009logoot} and
Treedoc~\cite{preguica2009commutative}, identifiers have a linear space
complexity compared to the number of insert operations. Furthermore, their size
heavily depends on the position of insertions. For example, performing
insertions repetitively at the beginning of a document leads to identifiers
with a size equals to the number of insert operations. Consequently, such
approaches require an additional re-balance protocol, like tombstone CRDTs.
However, their underlying allocation strategies can be replaced by
LSEQ~\cite{nedelec2013lseq} that provides a sub-linear upper-bound on the space
complexity regardless of the editing behaviour.

The LSEQ approach uses two antagonist allocation strategies. Although
paper~\cite{ahmed2011evaluating} argues about the importance of concurrency in
CRDTs experiments, the original paper of LSEQ does not consider this aspect. In
this paper, we showed on synthetic documents that its original configuration
does not scale in the number of users. The added value of synthetic documents
over real collaborative documents is the total flexibility on their parameters,
especially on the editing behaviour. Indeed, the number of available real
collaborative documents (including concurrency) is very limited. Moreover, such
experiments are costly to set up, and often biased by the nature of the task to
perform.


In this paper, we propose to replace the random strategy choice component of
LSEQ by a hash-based choice strategy to reach an implicit consensus on which
strategy to employ. This configuration called \NAME{}, does not require
additional computation and extends the sub-linear complexity bound of LSEQ from
a single user to multiple collaborators regardless of the latency. Thus,
\NAME{} constitutes a safe allocation strategy for sequences CRDTs, even in
concurrent cases.


%%% Local Variables: 
%%% mode: latex
%%% TeX-master: "../dchanges"
%%% End: 
