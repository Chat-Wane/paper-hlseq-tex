
\section{Introduction}

Distributed collaborative editors allow to distribute the work across space,
time and organization. Some trending editors such as Google
Docs~\cite{nichols1995high} use the Operational Transform
approach~\cite{sun1998operational,sun1998achieving}. However, an alternative
exists based on Conflict-free Replicated Data
Types~\cite{shapiro2011comprehensive,shapiro2011conflict}.  Compared to OT,
CRDTs are more decentralized and scale better.

The CRDTs belong to the optimistic
replication~\cite{saito2002replication,saito2005optimistic}
approach. Therefore, replicas involved in the collaboration are guaranteed of
eventually converging to an identical state. To provide convergence in a
replicated sequence, CRDTs use unique identifiers to link elements. When the
sequence is a document, the elements can be characters, lines, paragraphs, etc.


Currently, we distinguish two types of sequence CRDTs:
\begin{inparaenum}[(i)]
\item The tombstone
  CRDTs~\cite{ahmed2011evaluating,grishchenko2010deep,oster2006data,preguica2009commutative,roh2011replicated,weiss2007wooki,wu2010partial,Yu2012stringwise}
  use fixed-size identifiers. However, the delete operations only mark and hide
  elements to the users. As a consequence, these deleted elements permanently
  affect performances.
\item The variable-size identifiers
  CRDTs~\cite{preguica2009commutative,weiss2009logoot} directly encode the
  order relation in the identifiers. However, their identifiers can grow
  linearly depending on the editing behaviour. Consequently, they remain unsafe
  in the distributed collaborative editing context.
\end{inparaenum}

To overcome their respective issues, these approaches need an additional
protocol~\cite{letia2009crdts,roh2011replicated} related to garbage collection
mechanisms. Nevertheless, these protocols require global knowledge over
participants. Such knowledge remains prohibitively expensive in the context of
distributed network subject to churn without making strong assumptions.

Recently, LSEQ~\cite{nedelec2013lseq} allowed variable-size identifiers CRDTs
to get rid of this costly additional protocol by decreasing their space
complexity from linear to sub-linearly upper-bounded. Yet, LSEQ does not
guarantee a safe allocation. Indeed, it uses multiple antagonist strategies
which, without any coordination between collaborators, leads to an exponential
growth of identifiers.

The contributions of this paper include:
\begin{inparaenum}[(i)]
\item experiments that highlight the negative effect of multiple users edition
  on the size of LSEQ identifiers
\item experiments that highlight that an increasing latency does not have a
  negative impact on the size of LSEQ identifiers.
\item an improvement of LSEQ called \NAME{} that extends the single user
  sub-linear upper-bound on space complexity to any number of users, regardless
  of the latency, and without any additional cost.
\end{inparaenum} 

Ultimately, \NAME{} allows distributed collaborative editors to safely use
variable-size identifiers CRDTs.

Section~\ref{sec:background} details variable-size identifiers CRDTs for
sequences and the allocation strategy LSEQ. It also highlights the motivation
of this paper.  Section~\ref{sec:proposal} presents \NAME{} that combines LSEQ
and a hash strategy choice to overcome the LSEQ limitations.
Section~\ref{sec:experiments} shows the results of experiments over LSEQ and
\NAME{} including latency and multiple users. Finally,
Section~\ref{sec:relatedwork} reviews related works.


%%% Local Variables: 
%%% mode: latex
%%% TeX-master: "../dchanges"
%%% End: 
